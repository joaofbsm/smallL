\section{Código e Utilização}

Por ser muito extenso, preferimos não descrever todo o código do \textbf{tradutor} neste documento, disponibilizando-o no repositório mencionado na seção de introdução.

Para obter o código, basta clonar o repositório utilizando o comando:

\begin{lstlisting}
git clone https://github.com/joaofbsm/smallL.git
\end{lstlisting}

ou baixar o \texttt{.zip} disponibilizado ao clicar em \textbf{"Clone or download"} e depois em \textbf{"Download ZIP"} (na página do repositório).

Caso tenha optado pela segunda opção, basta descompactar e entrar na pasta descompactada.

\subsection{Traduzindo}
Para facilitar a utilização do \textbf{tradutor} foram criados dois scripts \textit{bash} que condensam as tarefas de compilar o código do \textit{front-end}  e executar o \textit{front-end} (necessário para geração das quadruplas).

\begin{itemize}
\item \texttt{compile.sh}: responsável por compilar as classes Java necessárias para o funcionamento do \textit{front end}.
\item \texttt{execute.sh}: reponsável por testar todas as entradas de teste (código na linguagem \textbf{SmallL}) disponibilizadas no diretório \texttt{tests}.
\end{itemize}

Para traduzir os códigos intermediários gerados, basta executar o script \texttt{translate.sh} (após ter executado os dois scripts citados acima).

Para criar um caso de teste, basta adicionar um arquivo \texttt{.txt}, contendo o teste desejado (em linguagem SmallL), no diretório \texttt{tests} presente no diretório raiz do \textit{front-end}.

Caso deseje rodar um teste em específico, cuja entrada já está em formato de código intermediário, basta rodar:

\begin{lstlisting}
// mude para o diretorio raiz do front end
cd /caminho_para_diretorio_raiz/smallL
cd code/translator
python3 translator.py codigo_intermediario
\end{lstlisting}

Tal comando produzirá na saída padrão o código traduzido na sintaxe Jasmin. Para rodar a sequência de scripts completa, siga os seguintes passos:

\begin{lstlisting}
// mude para o diretorio raiz do front end
cd /caminho_para_diretorio_raiz/smallL
// compila
./compile.sh
// executa front-end
./execute.sh
// traduz codigo intermediario
./translate.sh
\end{lstlisting}

A saída dos testes se encontra na pasta \texttt{outputs}. Os \texttt{.txt} gerados são referentes ao código intermediário gerado pelo \textit{front-end}. Dentro do diretório \texttt{outputs} há uma pasta denominada \texttt{translated} que contém os códigos em Jasmin (\texttt{.j}) gerados pelo \textbf{tradutor} e os binários gerados pelo \textbf{Krakatau} (arquivos \texttt{.class} dentro das pastas \texttt{nome\_teste-bin/}), a partir dos códigos gerados pelo tradutor.

Para uma melhor formatação da saída, é aconselhável rodar o \texttt{disassembler} do \textbf{Krakatau} nos arquivos \texttt{.class} gerados. Para tal, rode a seguinte linha de comando:

\begin{lstlisting}
\\ entre no diretorio com a saida binaria
cd caminho_para_smallL/outputs/translated/nome_teste-bin/
\\ rode o disassembler
python2.7 ../../../tools/Krakatau/disassemble.py Main.class
\end{lstlisting}

Tal comando produzirá um arquivo \texttt{Main.j} em sintaxe Jasmin, contendo o código produzido pelo tradutor em uma formatação mais clara e organizada.