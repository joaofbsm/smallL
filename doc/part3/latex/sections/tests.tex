\section{Testes}

A seguir são apresentados alguns testes que tentam englobar todas as possíveis instruções geradas em sintaxe Jasmin. A sequência de arquivos para cada teste é a seguinte:

\begin{enumerate}
\item arquivo em linguagem \textbf{SmallL}
\item arquivo com código intermediário gerado a partir de 1.
\item arquivo com código traduzido a partir de 2.
\item saída do programa após rodar o binário gerado.
\end{enumerate}

\subsection{Teste 1}

O teste abaixo visa testar as funcionalidades básicas da linguagem. Os prints adicionados no código intermediário visam obter o valor da variável i ao longo do primeiro \texttt{do-while}, bem como o valor da terceira posição do vetor a criado (contagem começando de zero) e o valor da quarta posição do vetor a ao longo do \texttt{while}.

\begin{lstlisting}[caption=\textbf{Arquivo do teste 1 em linguagem SmallL}]
{
	int i; int j; float[4] a;
	
	i = 1;
	j = 10;
	a[0] = 1;
	a[1] = 5;
	a[3] = 0;

	do i = i+1; while( i < j);

	a[2] = a[0] + a[1];
	
	i = 0;

	while ( i < j){
		i = i+1;
		a[3] = a[3] + 1;
	}

}
\end{lstlisting}

\begin{lstlisting}[caption=\textbf{Arquivo com código intermediário para o teste 1 produzido pelo front-end (adicionado de comandos print "na mão")}]
L1:	i = 1
L3:	j = 10
L4:	t1 = 0 * 8
    a [ t1 ] = 1
L5:	t2 = 1 * 8
    a [ t2 ] = 5
L6:	t3 = 3 * 8
    a [ t3 ] = 0
L7:	i = i + 1
    print i
L9:	if i < j goto L7
L8:	t4 = 2 * 8
    t5 = 0 * 8
    t6 = a [ t5 ]
    t7 = 1 * 8
    t8 = a [ t7 ]
    t9 = t6 + t8
    a [ t4 ] = t9
    print a [ t4 ]
L10:	i = 0
L11:	iffalse i < j goto L2
L12:	i = i + 1
L13:	t10 = 3 * 8
    t11 = 3 * 8
    t12 = a [ t11 ]
    t13 = t12 + 1
    a [ t10 ] = t13
    print a [ t10 ]
    goto L11
L2:

\end{lstlisting}

\begin{lstlisting}[caption=\textbf{Código para o teste 1 em sintaxe Jasmin produzido pelo tradutor implementado}]
.version 50 0 
.class public super Main 
.super java/lang/Object 

.method public <init> : ()V 
	.code stack 1 locals 1 
L0:		aload_0 
L1:		invokespecial Method java/lang/Object <init> ()V 
L4:		return 
L5:     
	.end code 
.end method 

.method public static main : ([Ljava/lang/String;)V
	.code stack 4 locals 33

L1:											 ldc2_w 1.0
		dstore 1
L3:											 ldc2_w 10.0
		dstore 3
L4:											 ldc2_w 0.0
		ldc2_w 8.0
		dmul
		dstore 5
		sipush 1000
		newarray double
		astore 7
		dload 5
		ldc2_w 8.0
		ddiv
		dstore 5
		aload 7
		dload 5
		d2i
		ldc2_w 1.0
		dastore
L5:											 ldc2_w 1.0
		ldc2_w 8.0
		dmul
		dstore 9
		dload 9
		ldc2_w 8.0
		ddiv
		dstore 9
		aload 7
		dload 9
		d2i
		ldc2_w 5.0
		dastore
L6:											 ldc2_w 3.0
		ldc2_w 8.0
		dmul
		dstore 11
		dload 11
		ldc2_w 8.0
		ddiv
		dstore 11
		aload 7
		dload 11
		d2i
		ldc2_w 0.0
		dastore
L7:											 dload 1
		ldc2_w 1.0
		dadd
		dstore 1
		getstatic Field java/lang/System out Ljava/io/PrintStream;
		ldc 'i: '
		invokevirtual Method java/io/PrintStream print (Ljava/lang/String;)V
		getstatic Field java/lang/System out Ljava/io/PrintStream;
		dload 1
		invokevirtual Method java/io/PrintStream println (D)V
L9:											 dload 1
		dload 3
		dcmpg
		iflt L7
L8:											 ldc2_w 2.0
		ldc2_w 8.0
		dmul
		dstore 13
		ldc2_w 0.0
		ldc2_w 8.0
		dmul
		dstore 15
		dload 15
		ldc2_w 8.0
		ddiv
		dstore 15
		aload 7
		dload 15
		d2i
		daload
		dstore 17
		ldc2_w 1.0
		ldc2_w 8.0
		dmul
		dstore 19
		dload 19
		ldc2_w 8.0
		ddiv
		dstore 19
		aload 7
		dload 19
		d2i
		daload
		dstore 21
		dload 17
		dload 21
		dadd
		dstore 23
		dload 13
		ldc2_w 8.0
		ddiv
		dstore 13
		aload 7
		dload 13
		d2i
		dload 23
		dastore
		getstatic Field java/lang/System out Ljava/io/PrintStream;
		ldc 'a[t4]: '
		invokevirtual Method java/io/PrintStream print (Ljava/lang/String;)V
		getstatic Field java/lang/System out Ljava/io/PrintStream;
		aload 7
		dload 13
		d2i
		daload
		invokevirtual Method java/io/PrintStream println (D)V
L10:											 ldc2_w 0.0
		dstore 1
L11:											 dload 1
		dload 3
		dcmpg
		ifge L2
L12:											 dload 1
		ldc2_w 1.0
		dadd
		dstore 1
L13:											 ldc2_w 3.0
		ldc2_w 8.0
		dmul
		dstore 25
		ldc2_w 3.0
		ldc2_w 8.0
		dmul
		dstore 27
		dload 27
		ldc2_w 8.0
		ddiv
		dstore 27
		aload 7
		dload 27
		d2i
		daload
		dstore 29
		dload 29
		ldc2_w 1.0
		dadd
		dstore 31
		dload 25
		ldc2_w 8.0
		ddiv
		dstore 25
		aload 7
		dload 25
		d2i
		dload 31
		dastore
		getstatic Field java/lang/System out Ljava/io/PrintStream;
		ldc 'a[t10]: '
		invokevirtual Method java/io/PrintStream print (Ljava/lang/String;)V
		getstatic Field java/lang/System out Ljava/io/PrintStream;
		aload 7
		dload 25
		d2i
		daload
		invokevirtual Method java/io/PrintStream println (D)V
		goto L11
L2:											 return

	.end code 
.end method 
.sourcefile 'Main.java' 
.end class
\end{lstlisting}

\begin{lstlisting}[caption=\textbf{Saída após rodar o binário}]
i: 2.0
i: 3.0
i: 4.0
i: 5.0
i: 6.0
i: 7.0
i: 8.0
i: 9.0
i: 10.0
a[t4]: 6.0
a[t10]: 1.0
a[t10]: 2.0
a[t10]: 3.0
a[t10]: 4.0
a[t10]: 5.0
a[t10]: 6.0
a[t10]: 7.0
a[t10]: 8.0
a[t10]: 9.0
a[t10]: 10.0
\end{lstlisting}

Como podemos observar, os valores relacionados estão corretos. Em relação ao print da posição do vetor, é possível identificar variáveis temporárias (t4 e t10) indexando os vetores. Tais variáveis foram criadas pelo tradutor em tempo de tradução, mas que indicam respectivamente a teceira e a quarta posição do vetor a.

Vale ressaltar aqui que, nesse caso, o valor das variáveis temporárias havia sido multiplicado por 8 para indexar no código intermediário. No entanto, isso não funcionaria na \textit{JVM}, e portanto, quando usado para indexar, o tradutor atualiza essa variável dividindo-na 8. Sendo assim, como o comando \texttt{print a [ t4 ]} vem após a variável temporária já ter sido usada como índice, ele é equivalente a \texttt{print a [ 2 ]} e não \texttt{print a [ 16 ]}. Portanto, para printarmos na tela as posições do array, devemos indexar normalmente, e não utilizando múltiplos de 8.


\subsection{Teste 2}

O teste a seguir visa abordar mais algumas funcionalidades básicas da linguagem. Os prints adicionados ao código intermediário tem o intuito de obter o valor de i e y ao longo do \texttt{while} e do valor da variável x antes do término do \texttt{while} (ao entrar na condição do if) e após o término.

\begin{lstlisting}[caption=\textbf{Arquivo para o teste 2 em linguagem SmallL}]
{
	int i; int k; float x; float y;

	i = 0;
	k = 10;
	x = 2;
	y = 0;

	while (i < k){

		if (i >= 5){
			x = x + 1;
			break;
		}

		i = i + 1;
		y = y + 1;
	}

	x = x + y;

}
\end{lstlisting}

\begin{lstlisting}[caption=\textbf{Arquivo com código intermediário para o teste 2 produzido pelo front-end (adicionado de comandos print "na mão")}]
L1:	i = 0
L3:	k = 10
L4:	x = 2
L5:	y = 0
L6:	iffalse i < k goto L7
L8:	iffalse i >= 5 goto L9
L10:	x = x + 1
     print x
L11:	goto L7
L9:	i = i + 1
    print i
L12:	y = y + 1
     print y
     goto L6
L7:	x = x + y
    print x
L2:
\end{lstlisting}

\begin{lstlisting}[caption=\textbf{Código em sintaxe Jasmin para o teste 2 produzido pelo tradutor implementado}]
.version 50 0 
.class public super Main 
.super java/lang/Object 

.method public <init> : ()V 
	.code stack 1 locals 1 
L0:		aload_0 
L1:		invokespecial Method java/lang/Object <init> ()V 
L4:		return 
L5:     
	.end code 
.end method 

.method public static main : ([Ljava/lang/String;)V
	.code stack 4 locals 9

L1:											 ldc2_w 0.0
		dstore 1
L3:											 ldc2_w 10.0
		dstore 3
L4:											 ldc2_w 2.0
		dstore 5
L5:											 ldc2_w 0.0
		dstore 7
L6:											 dload 1
		dload 3
		dcmpg
		ifge L7
L8:											 dload 1
		ldc2_w 5.0
		dcmpl
		iflt L9
L10:											 dload 5
		ldc2_w 1.0
		dadd
		dstore 5
		getstatic Field java/lang/System out Ljava/io/PrintStream;
		ldc 'x: '
		invokevirtual Method java/io/PrintStream print (Ljava/lang/String;)V
		getstatic Field java/lang/System out Ljava/io/PrintStream;
		dload 5
		invokevirtual Method java/io/PrintStream println (D)V
L11:											 goto L7
L9:											 dload 1
		ldc2_w 1.0
		dadd
		dstore 1
		getstatic Field java/lang/System out Ljava/io/PrintStream;
		ldc 'i: '
		invokevirtual Method java/io/PrintStream print (Ljava/lang/String;)V
		getstatic Field java/lang/System out Ljava/io/PrintStream;
		dload 1
		invokevirtual Method java/io/PrintStream println (D)V
L12:											 dload 7
		ldc2_w 1.0
		dadd
		dstore 7
		getstatic Field java/lang/System out Ljava/io/PrintStream;
		ldc 'y: '
		invokevirtual Method java/io/PrintStream print (Ljava/lang/String;)V
		getstatic Field java/lang/System out Ljava/io/PrintStream;
		dload 7
		invokevirtual Method java/io/PrintStream println (D)V
		goto L6
L7:											 dload 5
		dload 7
		dadd
		dstore 5
		getstatic Field java/lang/System out Ljava/io/PrintStream;
		ldc 'x: '
		invokevirtual Method java/io/PrintStream print (Ljava/lang/String;)V
		getstatic Field java/lang/System out Ljava/io/PrintStream;
		dload 5
		invokevirtual Method java/io/PrintStream println (D)V
L2:											 return

	.end code 
.end method 
.sourcefile 'Main.java' 
.end class


\end{lstlisting}

\begin{lstlisting}[caption=\textbf{Saída após rodar o binário}]
i: 1.0
y: 1.0
i: 2.0
y: 2.0
i: 3.0
y: 3.0
i: 4.0
y: 4.0
i: 5.0
y: 5.0
x: 3.0
x: 8.0
\end{lstlisting}

Como podemos observar, os valores estão de acordo com o esperado.

