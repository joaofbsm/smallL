\section{Introdução}

O trabalho em questão tem como objetivo apresentar o compilador integrado, desenvolvido ao longo dos trabalhos anteriores (\textit{front-end} e o tradutor). Para auxiliar a implementação(do tradutor) foram utilizadas as seguintes ferramentas:

\begin{itemize}
\item \textbf{Jasmin}\cite{trove.nla.gov.au/work/14968069}: sintaxe de bytecodes extendida.
\item \textbf{Krakatau}: decompiler, disassembler e assembler para arquivos \texttt{.class} Java.
\end{itemize}

O trabalho foi desenvolvido utilizando a ferramenta de versionamento Git juntamente com a plataforma de desenvolvimento remoto GitHub. O repositório do projeto contém não só o código fonte, mas também os scripts auxiliares desenvolvidos e os arquivos de teste, podendo ser acessado no endereço \url{https://github.com/joaofbsm/smallL}.

Esse relatório não visa explicar em detalhes cada componente do compilador, uma vez que isso já foi feito nos relatórios passados. Com isso, descreveremos como utilizar o compilador e rodar os programas gerados.