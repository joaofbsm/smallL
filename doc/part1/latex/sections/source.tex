\section{Código e Utilização}

\subsection{Obtendo o código fonte}
Por ser muito extenso, preferimos não descrever todo o código do compilador neste documento, disponibilizando-o no repositório mencionado na seção de introdução.

Para obter o código, basta clonar o repositório utilizando o comando:

\begin{lstlisting}
git clone https://github.com/joaofbsm/smallL.git
\end{lstlisting}

ou baixar o \texttt{.zip} disponibilizado ao clicar em \textbf{"Clone or download"} e depois em \textbf{"Download ZIP"} (na página do repositório).

Caso tenha optado pela segunda opção, basta descompactar e entrar na pasta descompactada.

\subsection{Compilando e executando}
Para facilitar a utilização do \textit{front end} foram criados dois scripts \textit{bash} que condensam as tarefas de compilar o código e executar testes em apenas duas chamadas na linha de comando.

\begin{itemize}
\item \texttt{compile.sh}: responsável por compilar as classes Java necessárias para o funcionamento do \textit{front end}.
\item \texttt{execute.sh}: reponsável por testar todas as entradas de teste disponibilizadas no diretório \texttt{tests}.
\end{itemize}

Para criar um caso de teste, basta adicionar um arquivo \texttt{.txt}, contendo o teste desejado, no diretório \texttt{tests} presente no diretório raiz do \textit{front end}

Para rodar os scripts, siga os seguintes passos:\\



\begin{lstlisting}
// mude para o diretorio raiz do front end
cd /caminho_para_diretorio_raiz/smallL
// compila
./compile.sh
// executa testes
./execute.sh
\end{lstlisting}

A saída dos testes é a padrão, ou seja, será escrita no terminal para cada um dos testes presentes no diretório.